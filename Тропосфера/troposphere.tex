\documentclass[a4paper,12pt]{article} % добавить leqno в [] для нумерации слева

%%% Работа с русским языком
\usepackage{cmap}					% поиск в PDF
\usepackage{mathtext} 				% русские буквы в фомулах
\usepackage[T2A]{fontenc}			% кодировка
\usepackage[utf8]{inputenc}			% кодировка исходного текста
\usepackage[english,russian]{babel}	% локализация и переносы
\usepackage[left=1cm,right=1cm,top=1cm,bottom=1cm,bindingoffset=0cm]{geometry}
\usepackage{graphicx}
\graphicspath{{img/}}
\DeclareGraphicsExtensions{.pdf,.png,.jpg}

%% Шрифты
\usepackage{euscript}	 % Шрифт Евклид
\usepackage{mathrsfs} % Красивый матшрифт

%%% Заголовок
\author{\LaTeX{} в Вышке}
\title{Гвоздарев}
\date{\today}

\begin{document} % конец преамбулы, начало документа

\begin{flushright}
\large{\textbf{Тропосфера}}
\end{flushright}

Тропосфера (от греч. trоpos — поворот, изменение и сфера), нижняя, преобладающая по массе часть земной атмосферы, в которой температура понижается с высотой. Тропосфера простирается в среднем до высот 8—10 км в полярных широтах, 10—12 км в умеренных, 16—18 км в тропических. Над тропосферой располагается стратосфера, от которой тропосфера отделена сравнительно тонким переходным слоем — \textbf{\textsl{тропопаузой}}. В тропосфере сосредоточено более всей массы атмосферного воздуха. Вся деятельность человека проходит в тропосфере. Самые высокие горы остаются в пределах тропосферы, даже воздушный транспорт лишь частично выходит за пределы тропосферы — в стратосферу. \

Газовый состав тропосферы постоянен и идентичен сос­таву у поверхности: 78\% азота, 21\% кислорода, 0,33\% аргона, 0,03\% CO2 и т. д. Содержание водяного пара - от 0 до 4\% по объ­ёму. \

Типов радиоволн, передающиеся через тропосферу, существует целое многообразие. Например, с помощью них передаются сигналы от различных видов связи (телевидение, телеграф, телефон) и сигналы, позволяющие управлять различными системами на расстоянии (радотелемеханика). \

Надёжная работа радиолиний обусловлена, во-первых, качественной работой приемо-передающих устройств; во-вторых, правильным выбором рабочей частоты, приемных и передающих антенн, и, в-третьих, условиями распространения волн. \textbf{\textsl{Мы сосредоточимся на последнем}}. \\

%Начнем с определения факторов, влияющих на распространение радиоволн в Т.:
%
%\begin{enumerate}
%\item дифракция радиоволн, т.е. огибание ими Земли
%\item рефракция (искривление) волн в тропосфере
%\item отражение от земной поверхности
%\item отражение от ионосферы
%\item поглощение энергии радиоволн газами и метеоосадками
%\end{enumerate}

В электрическом отношении тропосфера представляет собой весьма неоднородную среду, вследствие чего в ней происходит искривление (рефракция) траекторий радиоволн, а следовательно, изменение направления прихо­да волны и напряженности поля на данном расстоянии.

Проводимость тропосферы \(\sigma\) для частот, соответствующих радиоволнам (за исключением миллиметровых волн), практически равна \(0\); диэлектрическая проницаемость \(\epsilon\) и, следовательно, показатель преломления n являются функциями давления и температуры воздуха, а также давления водяного пара. У поверхности Земли \(n >> 1,0003\). Изменение \(\epsilon\) и \(n\) с высотой зависит от метеорологических условий. Обычно e и n уменьшаются, а фазовая скорость u растет с высотой. Это приводит к искривлению радиолучей. Если в тропосфере под углом к горизонту распространяется волна,  фронт (луч) которой совпадает с прямой AB, то вследствие того, что в верхних слоях тропосферы волна распространяется с большей скоростью, чем в нижних, верхняя часть фронта волны обгоняет нижнюю и фронт волны поворачивается (луч искривляется). Т. к. n с высотой убывает, то радиолучи отклоняются к Земле. 

\end{document} % конец документа